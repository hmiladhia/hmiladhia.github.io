%%%%%%%%%%%%%%%%%
% This is an example CV created using altacv.cls (v1.3, 10 May 2020) written by
% LianTze Lim (liantze@gmail.com), based on the
% Cv created by BusinessInsider at http://www.businessinsider.my/a-sample-resume-for-marissa-mayer-2016-7/?r=US&IR=T
%
%% It may be distributed and/or modified under the
%% conditions of the LaTeX Project Public License, either version 1.3
%% of this license or (at your option) any later version.
%% The latest version of this license is in
%%    http://www.latex-project.org/lppl.txt
%% and version 1.3 or later is part of all distributions of LaTeX
%% version 2003/12/01 or later.
%%%%%%%%%%%%%%%%

%% If you are using \orcid or academicons
%% icons, make sure you have the academicons
%% option here, and compile with XeLaTeX
%% or LuaLaTeX.
% \documentclass[10pt,a4paper,academicons]{altacv}

%% Use the "normalphoto" option if you want a normal photo instead of cropped to a circle

\documentclass[10pt,letter,ragged2e,withhyper]{altacv}

%% AltaCV uses the fontawesome5 and academicon fonts and packages.
%% See http://texdoc.net/pkg/fontawesome5 and http://texdoc.net/pkg/academicons for full list of symbols.
%% You MUST compile with XeLaTeX or LuaLaTeX if you want to use academicons.

% Change the page layout if you need to
\geometry{left=1cm,right=1cm,top=0.8cm,bottom=0.8cm,columnsep=0.75cm}

% The paracol package lets you typeset columns of text in parallel
\usepackage{paracol}


% Change the font if you want to, depending on whether
% you're using pdflatex or xelatex/lualatex
\ifxetexorluatex
  % If using xelatex or lualatex:
  \setmainfont{Lato}
\else
  % If using pdflatex:
  \usepackage[default]{lato}
\fi

% Change the colours if you want to
\definecolor{VividPurple}{HTML}{3E0097}
\definecolor{SlateGrey}{HTML}{2E2E2E}
\definecolor{LightGrey}{HTML}{666666}
\definecolor{LimeGreen}{HTML}{11710E}
\definecolor{Blue}{HTML}{253489}
\definecolor{Teal}{HTML}{109EAA}
\definecolor{Silver}{HTML}{B5B5B5}
\definecolor{DarkPastelRed}{HTML}{450808}
\definecolor{PastelRed}{HTML}{8F0D0D}
\definecolor{GoldenEarth}{HTML}{E7D192}
\definecolor{DarkGreen}{HTML}{303d36}
\definecolor{MyGreen}{HTML}{88A05B}

\colorlet{name}{black}
\colorlet{tagline}{teal}
\colorlet{heading}{teal}
\colorlet{headingrule}{teal}
\colorlet{subheading}{DarkGreen}
\colorlet{accent}{teal}
\colorlet{emphasis}{SlateGrey}
\colorlet{body}{LightGrey}

% Change some fonts, if necessary

% Change the bullets for itemize and rating marker
% for \cvskill if you want to
\renewcommand{\itemmarker}{{\small\textbullet}}
\renewcommand{\ratingmarker}{\faCircle}

\renewcommand{\divider}{\textcolor{body!30}{\hdashrule{\linewidth}{0.6pt}{0.5ex}}\medskip}



%% sample.bib contains your publications

\begin{document}
\newcommand{\cvpart}[1]{\ifstrequal{#1}{}{}{{\small\makebox[0.5\linewidth][l]{~#1}}}}

\name{Dhia Hmila}
\tagline{Data Scientist Senior chez AXA France}\photoR{3.5cm}{photo.png}
\photoL{3cm}{qrcode.png}

\personalinfo{%
  \email{dhia.hmila@dasquare.xyz}  \phone{+33610201606}  \location{92700, Colombes, France}  \homepage{hmiladhia.github.io}
    \printinfo{\faGithub}{hmiladhia}[https://github.com/hmiladhia]
    \printinfo{\faLinkedin}{dhia-hmila}[https://www.linkedin.com/in/dhia-hmila]
    \printinfo{\faMedium}{Dhia Hmila}[https://dhiahmila.medium.com/]
    \printinfo{\faStackOverflow}{dhia-hmila}[https://stackoverflow.com/users/8655480/dhia-hmila?tab=profile]
  
  % \printinfo{\faPaw}{Hey ho!}
}

\makecvheader

%% Depending on your tastes, you may want to make fonts of itemize environments slightly smaller
\AtBeginEnvironment{itemize}{\small}

%% Set the left/right column width ratio to 6:4.
\columnratio{0.6}

% Start a 2-column paracol. Both the left and right columns will automatically
% break across pages if things get too long.
\begin{paracol}{2}

\cvsection{Expérience Professionnelle}

\cvevent{Data Scientist Senior}{Équipe Data Science \& IA, AXA France}{        
      Juin 2021 - Aujourd'hui
      }{Nanterre, France}

\begin{itemize}
  \item Développé un système RAG basé sur un LLM pour répondre aux questions des agents d'assurance et réduire la charge du service support.
  \item Animé des ateliers avec des ingénieurs ML, des ingénieurs de données et des data scientists pour développer le Guide des Meilleures Pratiques MLOps pour AXA France.
  \item Créé un template Cookiecutter pour de nouveaux projets ML afin d'accélérer le time-to-market (TTM) chez AXA.
  \item Construit des systèmes d'IA pour l'extraction d'informations structurées à partir de documents (CNI, permis, etc.), permettant le traitement de plus de 6 millions de documents avec PySpark.
  \item Conçu des packages de document processing réutilisables (dont certains en open-source) pour accélérer le développement de moteurs de document processing.
  \item Automatisé l'indexation de 70 \% des documents entrants d'AXA France en utilisant des systèmes d'IA réentraînables pour la classification de documents.
  \item Réduction de la charge de travail du service d'assistance en développant un système RAG basé sur un LLM pour répondre aux questions des agents d'assurance.
  \item Construit des pipelines Azure ML réutilisables pour l'entraînement continu (CT), facilitant le déploiement des modèles dans le Model Registry.
  \item Conçu des templates Azure DevOps de CI/CD, améliorant la qualité du code et le temps de livraison des projets.
  \item Coordonné le processus d'annotation avec les métiers pour garantir des ensembles de données de haute qualité à travers divers projets.
  \item Dirigé la COP Python d'AXA en organisant des présentations mensuelles sur des sujets tels que les tests, le web scraping et la qualité du code.
  \item Animé des sessions de formation pour la manipulation de données avec Python pour plus de 30 collaborateurs d'AXA, renforçant ainsi leurs compétences
\end{itemize}

\divider
\cvevent{Graduate Program - Data Scientist}{Équipe Data, Fraud, Waste \& Abuse, AXA France}{        
      Mai 2020 - Juin 2021
      }{Nanterre, France}

\begin{itemize}
  \item Construit un outil de détection de fraudes et d'abus piloté par l'IA, permettant des économies dépassant 150 000 € en 2022.
  \item Conçu et entraîné plusieurs modèles de détection de fraude, améliorant significativement les performances de la détection.
\end{itemize}

\divider
\cvevent{Data Scientist}{Shift Technology}{        
      Avril 2019 - Mars 2020
      }{Paris, France}

\begin{itemize}
  \item Appliqué des techniques de NLP pour cluster-iser les notes des gestionnaires de sinistres, révélant des patterns de fraude émergents grâce à une visualisation avancée.
  \item Développé et déployé de nouveaux modèles d'IA pour la solution de détection de fraude de Shift, améliorant ainsi les offres aux clients.
\end{itemize}

\divider
\cvevent{Stage de recherche}{U2IS - ENSTA Paris}{        
      Mai 2018 - Août 2018
      }{Palaiseau, France}

\begin{itemize}
  \item Créé un ensemble de données d'images synthétiques en simulant un agent dans un environnement interne, en utilisant le ray tracing pour un rendu d'image réaliste.
  \item Exploré l'oubli catastrophique dans l'apprentissage incrémental des composantes principales des images générées.
\end{itemize}



\cvsection{Projets}

\cvevent{\faCube \href{https://github.com/hmiladhia/mlflower}{mlflower: Outil d'orchestration pour les projets MLflow.}}{\cvtag{mlflow}\cvtag{orchestration}\cvtag{experiment-tracking}\cvtag{open-source}}{        
      Novembre 2023 - Aujourd'hui
      }{}

Un outil d'orchestration qui améliore les projets MLflow en fournissant une gestion intégrée des workflows multi-étapes.

\begin{itemize}
      \item Utilisé des graphes orientés acyclique (DAG) pour gérer des workflows ML complexes, garantissant une exécution logique des tâches.
      \item Implémenté un tri topologique pour optimiser l'ordre des tâches en fonction des dépendances, améliorant ainsi l'efficacité des workflows.
      \item Amélioré les projets MLflow avec une gestion claire des dépendances pour des expériences en plusieurs étapes.
      \item Intégré des outils de visualisation pour une compréhension intuitive des workflows et des dépendances.
  \end{itemize}

\divider
\cvevent{\faCube \href{https://github.com/hmiladhia/pysira}{pysira: publication de CVs dans différents formats.}}{\cvtag{Github Actions}\cvtag{Jinja2 Templates}\cvtag{jsonschema}}{        
      Février 2023 - Aujourd'hui
      }{}

Un outil pour exporter des fichiers 'jsonresume' vers divers formats (HTML, TeX, PDF) et langues.

\begin{itemize}
      \item Développé un pipeline CI/CD utilisant GitHub Actions pour automatiser la publication de CV dans plusieurs formats et thèmes.
      \item Permis l'exportation de fichiers 'jsonresume' vers HTML, TeX, PDF et d'autres formats.
      \item Démontré la fonctionnalité avec un CV en direct : hmiladhia.github.io, hmiladhia.github.io/cv.pdf.
  \end{itemize}

\divider
\cvevent{\faCube \href{https://github.com/AxaFrance/axa-fr-splitter}{Splitter: Outil de Traitement de Documents}}{\cvtag{Traitement de Documents}\cvtag{RAG}\cvtag{open-source}}{        
      Décembre 2021 - Aujourd'hui
      }{}

Un package open-source offrant des outils pour traiter divers types de documents (PDF, TIFF, etc.) en images et extraire du texte en utilisant Python.

\begin{itemize}
      \item Prend en charge plusieurs formats de documents pour un traitement polyvalent.
      \item Architecture personnalisable avec support de plugins pour des fonctionnalités étendues.
  \end{itemize}

\divider
\cvevent{\faLaptop \href{https://spot-language.onrender.com/}{Spot-Language: Prédiction des langage de programmation}}{\cvtag{Classification}\cvtag{Flask}\cvtag{NLP}\cvtag{LIME}}{        
    Janvier 2020
  }{}

Un modèle de classification pour détecter les langages de programmation dans des extraits de code.

\begin{itemize}
      \item Construit un jeu de données d'entrainement provenant de repositories github publiques.
      \item Construit une expérience ML pour entraîner un modèle de classification pour la détection de langages de programmation.
      \item Amélioré l'interprétabilité du modèle en utilisant LIME pour de meilleures insights sur les prédictions.
      \item Déployé une application web de démonstration utilisant Flask pour présenter le modèle ML.
  \end{itemize}



%%%%%%%%%%%%%%%%%%%%%%%%%%%%%%%%%%%%%%%%%%%%%%%%%%%%%%%%%
%% Switch to the right column. This will now automatically move to the second
%% page if the content is too long.
\switchcolumn
\cvsection{A Propos De Moi}
{\small
Je suis un Data Scientist Senior spécialisé dans la détection de fraudes/anomalies 
et la classification de documents. J'ai travaillé sur une panoplie de sujets ML 
dont des sujets Computer Vision ou encore NLP/NLU.
Je possède également une solide expérience autour des pratiques de MLOps,
pour garantir une intégration fluide des modèles dans les environnements de production.
}

\cvsection{Cursus}

\cvevent{Double diplôme d'ingénieur}{\cvpart{\faUserGraduate ENSTA ParisTech }\cvpart{\faUserGraduate  ENIT}}{2016 - 2019}{France - Tunisia}
{\small Spécialisé en \textbf{Intelligence Artificielle}}

\divider
\cvevent{Cycle préparatoire}{\cvpart{\faUserGraduate IPEIT }\cvpart{\faUserGraduate  Esprit Prépa}}{2014 - 2015}{France - Tunisia}
{\small Spécialisé en \textbf{Math \& Physique}}
{\small Rang 10 au concours d'entrée aux écoles d'ingénieurs en Tunisie parmi plus de 3000 candidats.}



\cvsection{Compétences}

\cvsubsection{Programmation}
\cvtag{\faPython Python}
\cvtag{ Rust}
\cvtag{\faJs JavaScript}
\cvtag{ C/C++}

\divider
\cvsubsection{Intelligence Artificielle}
\cvtag{ scikit-learn}
\cvtag{ TensorFlow}
\cvtag{ PyTorch}
\cvtag{ Transformers}
\cvtag{ Generative AI}

\divider
\cvsubsection{MLOps}
\cvtag{ MLflow}
\cvtag{\faGithub Github Actions}
\cvtag{\faMicrosoft Azure Pipelines}
\cvtag{\faDocker Docker}
\cvtag{ Kubernetes}

\divider
\cvsubsection{Natural Language Processing}
\cvtag{ spaCy}
\cvtag{ Langchain}
\cvtag{ Pydantic-ai}
\cvtag{ RAG}
\cvtag{ LLMs}
\cvtag{ ReACT}

\divider
\cvsubsection{Data Wrangling}
\cvtag{ pandas}
\cvtag{ Polars}
\cvtag{ PySpark}
\cvtag{\faDatabase SQL}

\divider
\cvsubsection{Computer Vision}
\cvtag{ OpenCV}
\cvtag{ Vision Transformers}
\cvtag{ YOLO}

\divider
\cvsubsection{Cloud}
\cvtag{\faMicrosoft Azure ML}
\cvtag{ Databricks}



\cvsection{Langues}

\cvskill{\large Arabe}{5}
\cvskill{\large Anglais}{5}
\cvskill{\large Français}{5}


\cvsection{Certifications}

\cvachievement{\faGraduationCap}{\href{https://www.coursera.org/account/accomplishments/specialization/6XMVEAR45K6G}{Machine Learning Engineering for Production (MLOps) Specialization}}{\cvpart{DeepLearning.AI}\cvpart{\faCalendar 2022 2022}}
\cvachievement{\faGraduationCap}{\href{https://www.coursera.org/account/accomplishments/specialization/G8MVU96LKELW}{Natural Language Processing Specialization}}{\cvpart{DeepLearning.AI}\cvpart{\faCalendar 2024 2024}}
\cvachievement{\faGraduationCap}{\href{https://www.coursera.org/account/accomplishments/specialization/J5FL39Y2CK4D}{Cloud-Native Development with OpenShift and Kubernetes Specialization}}{\cvpart{Red Hat}\cvpart{\faCalendar 2023 2023}}
\cvachievement{\faGraduationCap}{\href{https://www.coursera.org/account/accomplishments/certificate/ABFVM3MXBCJN}{Rust Fundamentals}}{\cvpart{Duke University}\cvpart{\faCalendar 2024 2024}}
\cvachievement{\faGraduationCap}{\href{https://www.coursera.org/account/accomplishments/certificate/XUB5Z5U3HF9C}{Rust for DevOps}}{\cvpart{Duke University}\cvpart{\faCalendar 2024 2024}}
\cvachievement{\faGraduationCap}{\href{https://www.coursera.org/account/accomplishments/certificate/4WPJZL5DER96}{Rust for LLMOps (Large Language Model Operations)}}{\cvpart{Duke University}\cvpart{\faCalendar 2024 2024}}
\cvachievement{\faGraduationCap}{\href{https://www.coursera.org/account/accomplishments/certificate/7X42H7MNSLC3}{DevOps, DataOps, MLOps}}{\cvpart{Duke University}\cvpart{\faCalendar 2024 2024}}
\cvachievement{\faGraduationCap}{\href{https://www.coursera.org/account/accomplishments/records/E9F3SJLMEFDL}{MLOps Tools: MLflow and Hugging Face}}{\cvpart{Duke University}\cvpart{\faCalendar 2023 2023}}
\cvachievement{\faGraduationCap}{\href{https://www.coursera.org/account/accomplishments/records/6ZK4EC55RSC6}{Introduction to Large Language Models}}{\cvpart{Google Cloud}\cvpart{\faCalendar 2023 2023}}
\cvachievement{\faGraduationCap}{\href{}{TOEIC Listening and Reading Test (Score 985/990)}}{\cvpart{ETS Global}\cvpart{\faCalendar 2018 2018}}

\end{paracol}

\end{document}