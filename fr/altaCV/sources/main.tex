%%%%%%%%%%%%%%%%%
% This is an example CV created using altacv.cls (v1.3, 10 May 2020) written by
% LianTze Lim (liantze@gmail.com), based on the
% Cv created by BusinessInsider at http://www.businessinsider.my/a-sample-resume-for-marissa-mayer-2016-7/?r=US&IR=T
%
%% It may be distributed and/or modified under the
%% conditions of the LaTeX Project Public License, either version 1.3
%% of this license or (at your option) any later version.
%% The latest version of this license is in
%%    http://www.latex-project.org/lppl.txt
%% and version 1.3 or later is part of all distributions of LaTeX
%% version 2003/12/01 or later.
%%%%%%%%%%%%%%%%

%% If you are using \orcid or academicons
%% icons, make sure you have the academicons
%% option here, and compile with XeLaTeX
%% or LuaLaTeX.
% \documentclass[10pt,a4paper,academicons]{altacv}

%% Use the "normalphoto" option if you want a normal photo instead of cropped to a circle

\documentclass[10pt,letter,ragged2e,withhyper]{altacv}

%% AltaCV uses the fontawesome5 and academicon fonts and packages.
%% See http://texdoc.net/pkg/fontawesome5 and http://texdoc.net/pkg/academicons for full list of symbols.
%% You MUST compile with XeLaTeX or LuaLaTeX if you want to use academicons.

% Change the page layout if you need to
\geometry{left=1cm,right=1cm,top=0.8cm,bottom=0.8cm,columnsep=0.75cm}

% The paracol package lets you typeset columns of text in parallel
\usepackage{paracol}


% Change the font if you want to, depending on whether
% you're using pdflatex or xelatex/lualatex
\ifxetexorluatex
  % If using xelatex or lualatex:
  \setmainfont{Lato}
\else
  % If using pdflatex:
  \usepackage[default]{lato}
\fi

% Change the colours if you want to
\definecolor{VividPurple}{HTML}{3E0097}
\definecolor{SlateGrey}{HTML}{2E2E2E}
\definecolor{LightGrey}{HTML}{666666}
\definecolor{LimeGreen}{HTML}{11710E}
\definecolor{Blue}{HTML}{253489}
\definecolor{Teal}{HTML}{109EAA}
\definecolor{Silver}{HTML}{B5B5B5}
\definecolor{DarkPastelRed}{HTML}{450808}
\definecolor{PastelRed}{HTML}{8F0D0D}
\definecolor{GoldenEarth}{HTML}{E7D192}
\definecolor{DarkGreen}{HTML}{303d36}
\definecolor{MyGreen}{HTML}{88A05B}

\colorlet{name}{black}
\colorlet{tagline}{teal}
\colorlet{heading}{teal}
\colorlet{headingrule}{teal}
\colorlet{subheading}{DarkGreen}
\colorlet{accent}{teal}
\colorlet{emphasis}{SlateGrey}
\colorlet{body}{LightGrey}

% Change some fonts, if necessary

% Change the bullets for itemize and rating marker
% for \cvskill if you want to
\renewcommand{\itemmarker}{{\small\textbullet}}
\renewcommand{\ratingmarker}{\faCircle}

\renewcommand{\divider}{\textcolor{body!30}{\hdashrule{\linewidth}{0.6pt}{0.5ex}}\medskip}



%% sample.bib contains your publications

\begin{document}
\newcommand{\cvpart}[1]{\ifstrequal{#1}{}{}{{\small\makebox[0.5\linewidth][l]{~#1}}}}

\name{Dhia Hmila}
\tagline{Data Scientist Senior chez AXA France}\photoR{3.5cm}{photo_light.png}
\photoL{3cm}{qrcode.png}

\personalinfo{%
  \email{dhiahmila@gmail.com}  \phone{+33610201606}  \location{92700, Colombes, France}  \homepage{hmiladhia.github.io}
    \printinfo{\faGithub}{hmiladhia}[https://github.com/hmiladhia]
    \printinfo{\faLinkedin}{dhia-hmila}[https://www.linkedin.com/in/dhia-hmila]
    \printinfo{\faMedium}{Dhia Hmila}[https://dhiahmila.medium.com/]
    \printinfo{\faStackOverflow}{dhia-hmila}[https://stackoverflow.com/users/8655480/dhia-hmila?tab=profile]
  
  % \printinfo{\faPaw}{Hey ho!}
}

\makecvheader

%% Depending on your tastes, you may want to make fonts of itemize environments slightly smaller
\AtBeginEnvironment{itemize}{\small}

%% Set the left/right column width ratio to 6:4.
\columnratio{0.6}

% Start a 2-column paracol. Both the left and right columns will automatically
% break across pages if things get too long.
\begin{paracol}{2}

\cvsection{Expérience Professionnelle}

\cvevent{Data Scientist Senior}{Équipe Data Science \& IA, AXA France}{        
      Juin 2021 - Aujourd'hui
      }{Nanterre, France}

\begin{itemize}
  \item Développement de modèles de classification de documents basé sur des technologies de Computer Vision (keras, Vision Transformer, openCV, ...) pour indexer les document de la GED ( Gestion éléctronique des documents ) d'AXA.
  \item Construction de Pipelines Azure ML réutilisables pour réentrainer automatiquement un système IA (Yolov5, tesseract, VGG16, Vision Transformer, ...) utilisé pour l'extraction d'informations (Permis, CNI, ...)
  \item Construction de Templates CI (Continuous Integration), CD (Continuous Deployment) \& CT (Continuous Training) pour les projets de l'équipe
  \item Responsable de la "COP Python" d'AXA et animateur d'événements mensuels de programmation ( testing, Web Scraping, packaging, qualité de code)
  \item Animation d'une formation Python et Manipulation de données en utilisant pandas \& pyspark pour des collaborateurs AXA
\end{itemize}

\divider
\cvevent{Graduate Program - Data Scientist}{Équipe Data, Fraud, Waste \& Abuse, AXA France}{        
      Mai 2020 - Juin 2021
      }{Nanterre, France}

\begin{itemize}
  \item Développement d'un outil de détection automatique de fraudes qui agit sur différents périmètres (Prestataires de santé, Bénéficiaires, Procédures médicales, ...) et déploiement sur Databricks.
  \item Création de différents modèles de détection de fraude pour alimenter l'outil de détection FWA.
\end{itemize}

\divider
\cvevent{Data Scientist}{Shift Technology}{        
      Avril 2019 - Mars 2020
      }{Paris, France}

\begin{itemize}
  \item Utilisation de technologies NLP pour le clustering des notes de gestionnaires de sinistres et création de nouveaux modèles de détection de fraude
  \item Amélioration et déploiement de la solution de détection de fraude Force de SHIFT pour un nouveau client
\end{itemize}



\cvsection{Projets}

\cvevent{\faCube \href{https://pypi.org/project/nbmanips/}{nbmanips: Découper, fusionner et convertir vos Notebooks}}{\cvtag{Jupyter}\cvtag{Databricks}\cvtag{Zeppelin}}{        
      Mai 2021 - Aujourd'hui
      }{}

Un package open-source d'outils de manipulation de Notebooks via des scripts python ou une CLI.


\divider
\cvevent{\faLaptop \href{https://spot-language.onrender.com/}{Spot-Language: Prédiction du language de programmation}}{\cvtag{Classification}\cvtag{Flask}\cvtag{NLP}\cvtag{Compression de Modèles}}{        
    Janvier 2020
  }{}

Spot-Language est un modèle de classification des languages utilisés dans des sources de code.

\begin{itemize}
      \item Construction de dataset d'entrainement provenant de repositories github publiques.
      \item Déploiement d'une application Web pour servir de démo au modèle IA.
  \end{itemize}

\divider
\cvevent{\faCube \href{https://pypi.org/project/Dmail/}{Dmail: Envoi d'e-mails au format Markdown}}{\cvtag{email}\cvtag{markdown}\cvtag{downloads\textgreater{}100k}}{        
      Avril 2020 - Mai 2020
      }{}

Un package python qui permet d'envoyer des e-mails au format markdown




%%%%%%%%%%%%%%%%%%%%%%%%%%%%%%%%%%%%%%%%%%%%%%%%%%%%%%%%%
%% Switch to the right column. This will now automatically move to the second
%% page if the content is too long.
\switchcolumn

\cvsection{Cursus}

\cvevent{Double diplôme d'ingénieur}{\cvpart{\faUserGraduate ENSTA ParisTech }\cvpart{\faUserGraduate  ENIT}}{2016 - 2019}{France - Tunisia}
{\small Spécialisé en \textbf{Intelligence Artificielle}}

\divider
\cvevent{Cycle préparatoire}{\cvpart{\faUserGraduate IPEIT }\cvpart{\faUserGraduate  Esprit Prépa}}{2014 - 2015}{France - Tunisia}
{\small Spécialisé en \textbf{Math \& Physique}}
{\small Rang 10 au concours d'entrée aux écoles d'ingénieurs en Tunisie parmi plus de 3000 candidats.}



\cvsection{Compétences}

\cvsubsection{Intelligence Artificielle}
\cvtag{ Anomaly Detection}
\cvtag{ Tensorflow}
\cvtag{ TFX}
\cvtag{ scikit-learn}
\cvtag{ NLP}
\cvtag{ nltk}
\cvtag{ spaCy}
\cvtag{ Gensim}
\cvtag{ pyod}

\divider
\cvsubsection{Data Wrangling}
\cvtag{ pandas}
\cvtag{ pyspark}
\cvtag{\faDatabase SQL}
\cvtag{ ETL}

\divider
\cvsubsection{Languages de programmation}
\cvtag{\faPython Python}
\cvtag{ C/C++}
\cvtag{ C\#}
\cvtag{\faJs JavaScript}
\cvtag{ Rust}

\divider
\cvsubsection{MLOps}
\cvtag{\faGithub Github Workflows}
\cvtag{\faMicrosoft Azure Pipelines}
\cvtag{\faGitSquare Git}
\cvtag{\faLinux Linux}
\cvtag{\faDocker Docker}
\cvtag{\faMicrosoft Azure ML}
\cvtag{ Databricks}

\divider
\cvsubsection{Web Scraping}
\cvtag{ selenium}
\cvtag{ requests}
\cvtag{ BeautifulSoup4}



\cvsection{Langues}

\cvskill{\large Arabe}{5}
\cvskill{\large Anglais}{5}
\cvskill{\large Français}{5}


\cvsection{Certifications}

\cvachievement{\faGraduationCap}{\href{https://www.coursera.org/account/accomplishments/specialization/6XMVEAR45K6G}{Machine Learning Engineering for Production (MLOps)}}{\cvpart{DeepLearning.AI}\cvpart{\faCalendar 2022 2022}}
\cvachievement{\faGraduationCap}{\href{https://www.coursera.org/account/accomplishments/specialization/G8MVU96LKELW}{Natural Language Processing}}{\cvpart{DeepLearning.AI}\cvpart{\faCalendar 2024 2024}}
\cvachievement{\faGraduationCap}{\href{https://www.coursera.org/account/accomplishments/specialization/J5FL39Y2CK4D}{Cloud-Native Development with OpenShift and Kubernetes Specialization}}{\cvpart{DeepLearning.AI}\cvpart{\faCalendar 2023 2023}}
\cvachievement{\faGraduationCap}{\href{https://www.coursera.org/account/accomplishments/records/E9F3SJLMEFDL}{MLOps Tools: MLflow and Hugging Face}}{\cvpart{Duke University}\cvpart{\faCalendar 2023 2023}}
\cvachievement{\faGraduationCap}{\href{https://www.coursera.org/account/accomplishments/records/6ZK4EC55RSC6}{Introduction to Large Language Models}}{\cvpart{Google Cloud}\cvpart{\faCalendar 2023 2023}}
\cvachievement{\faGraduationCap}{\href{}{TOEIC Listening and Reading Test (Score 985/990)}}{\cvpart{ETS Global}\cvpart{\faCalendar 2018 2018}}

\end{paracol}

\end{document}