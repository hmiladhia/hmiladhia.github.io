%%%%%%%%%%%%%%%%%
% This is an example CV created using altacv.cls (v1.3, 10 May 2020) written by
% LianTze Lim (liantze@gmail.com), based on the
% Cv created by BusinessInsider at http://www.businessinsider.my/a-sample-resume-for-marissa-mayer-2016-7/?r=US&IR=T
%
%% It may be distributed and/or modified under the
%% conditions of the LaTeX Project Public License, either version 1.3
%% of this license or (at your option) any later version.
%% The latest version of this license is in
%%    http://www.latex-project.org/lppl.txt
%% and version 1.3 or later is part of all distributions of LaTeX
%% version 2003/12/01 or later.
%%%%%%%%%%%%%%%%

%% If you are using \orcid or academicons
%% icons, make sure you have the academicons
%% option here, and compile with XeLaTeX
%% or LuaLaTeX.
% \documentclass[10pt,a4paper,academicons]{altacv}

%% Use the "normalphoto" option if you want a normal photo instead of cropped to a circle

\documentclass[10pt,letter,ragged2e,withhyper]{altacv}

%% AltaCV uses the fontawesome5 and academicon fonts and packages.
%% See http://texdoc.net/pkg/fontawesome5 and http://texdoc.net/pkg/academicons for full list of symbols.
%% You MUST compile with XeLaTeX or LuaLaTeX if you want to use academicons.

% Change the page layout if you need to
\geometry{left=1cm,right=1cm,top=1cm,bottom=1cm,columnsep=0.75cm}

% The paracol package lets you typeset columns of text in parallel
\usepackage{paracol}


% Change the font if you want to, depending on whether
% you're using pdflatex or xelatex/lualatex
\ifxetexorluatex
  % If using xelatex or lualatex:
  \setmainfont{Lato}
\else
  % If using pdflatex:
  \usepackage[default]{lato}
\fi

% Change the colours if you want to
\definecolor{VividPurple}{HTML}{3E0097}
\definecolor{SlateGrey}{HTML}{2E2E2E}
\definecolor{LightGrey}{HTML}{666666}
\definecolor{LimeGreen}{HTML}{11710E}
\definecolor{Blue}{HTML}{253489}
\definecolor{Teal}{HTML}{109EAA}
\definecolor{Silver}{HTML}{B5B5B5}
\definecolor{DarkPastelRed}{HTML}{450808}
\definecolor{PastelRed}{HTML}{8F0D0D}
\definecolor{GoldenEarth}{HTML}{E7D192}
\definecolor{DarkGreen}{HTML}{303d36}
\definecolor{MyGreen}{HTML}{88A05B}

\colorlet{name}{black}
\colorlet{tagline}{teal}
\colorlet{heading}{teal}
\colorlet{headingrule}{teal}
\colorlet{subheading}{DarkGreen}
\colorlet{accent}{teal}
\colorlet{emphasis}{SlateGrey}
\colorlet{body}{LightGrey}

% Change some fonts, if necessary

% Change the bullets for itemize and rating marker
% for \cvskill if you want to
\renewcommand{\itemmarker}{{\small\textbullet}}
\renewcommand{\ratingmarker}{\faCircle}

\renewcommand{\divider}{\textcolor{body!30}{\hdashrule{\linewidth}{0.6pt}{0.5ex}}\medskip}



%% sample.bib contains your publications

\begin{document}
\newcommand{\cvpart}[1]{\ifstrequal{#1}{}{}{{\small\makebox[0.5\linewidth][l]{~#1}}}}

\name{Dhia Hmila}
\tagline{Senior Data Scientist at AXA France}
%% You can add multiple photos on the left or right
\photoR{3.5cm}{photo_light.png}
%\photoL{2cm}{Yacht_High,Suitcase_High}
\personalinfo{%
  % Not all of these are required!
  % You can add your own with \printinfo{symbol}{detail}
  \email{dhiahmila@gmail.com}  \phone{+33610201606}  \location{92700, Colombes, France}  \homepage{hmiladhia.github.io}
    \printinfo{\faGithub}{hmiladhia}[https://github.com/hmiladhia]
    \printinfo{\faLinkedin}{dhia-hmila}[https://www.linkedin.com/in/dhia-hmila]
    \printinfo{\faMedium}{Dhia Hmila}[https://dhiahmila.medium.com/]
    \printinfo{\faStackOverflow}{dhia-hmila}[https://stackoverflow.com/users/8655480/dhia-hmila?tab=profile]
  
  % \printinfo{\faPaw}{Hey ho!}
  %% Or you can declare your own field with
  % \NewInfoFiled{fieldname}{symbol}[optional hyperlink prefix] and use it:
  % \NewInfoField{gitlab}{\faGitlab}[https://gitlab.com/]
  % \gitlab{your_id}
}

\makecvheader

%% Depending on your tastes, you may want to make fonts of itemize environments slightly smaller
\AtBeginEnvironment{itemize}{\small}

%% Set the left/right column width ratio to 6:4.
\columnratio{0.6}

% Start a 2-column paracol. Both the left and right columns will automatically
% break across pages if things get too long.
\begin{paracol}{2}

\cvsection{Professional experience}

\cvevent{Senior Data Scientist}{Data Science Team, AXA France}{        
      June 2021 - Present
      }{Nanterre, France}

\begin{itemize}
  \item Development of document classification models using Computer Vision (Vision Transformer, VGG16, openCV, ...) \& OCR (tesseract, EasyOCR, ...) to automatically index AXA's Document management system ( GED ).
  \item Development of text extraction AI Systems using Computer Vision (Yolov5, tesseract, VGG16, Vision Transformer, ...) to read documents (Driver's license, National ID card, Passeports, "Relevé d'information" ...)
  \item Development of resuable assets (packages, models, micro-services, ...) to use across the Team's projects.
  \item Building reusable Azure ML Pipelines (Continuous Training) to automatically retrain AI Systems \& publish to Model Registery.
  \item Defining CI (Continuous Integration), CD (Continuous Deployment) \& CT (Continuous Training) Templates for the Team's projects.
  \item Leading AXA's "Python COP" and organising a series of monthly programming talks/events (e.i. testing, Web Scraping, packaging, code quality, ...)
  \item Leading a series of training sessions/courses on Python, Data Manipulation (pandas \& pyspark), Software Engineering best practices for AXA collaborators
\end{itemize}

\divider
\cvevent{Graduate Data Scientist}{Data, Fraud, Waste \& Abuse Team, AXA France}{        
      May 2020 - June 2021
      }{Nanterre, France}

\begin{itemize}
  \item Development of an AI-based Fraud, Waste, and Abuse Detection Tool that acts on various scopes (Claims, Health Providers, Beneficiaries, ...) \& Deployment on Databricks
  \item Creation of various fraud detection models for fraudulent patterns as part of the FWA detection tool.
\end{itemize}

\divider
\cvevent{Data Scientist}{Shift Technology}{        
      April 2019 - March 2020
      }{Paris, France}

\begin{itemize}
  \item Use of NLP technologies for clustering Claim managers' notes and visualizing results for detecting new patterns of fraud
  \item Build AI Models for Shift's fraud detection solution \& Deployment for a new customer
\end{itemize}



\cvsection{Projects}

\cvevent{\faCube \href{https://github.com/hmiladhia/pysira}{pysira: publish resumes in different formats}}{\cvtag{Github Actions}\cvtag{Jinja2 Templates}\cvtag{jsonschema}}{        
      February 2023 - Present
      }{}

CLI tool to export "jsonresume" files to different formats (html, tex, pdf, ...) and different languages.

\begin{itemize}
      \item Creation of a CI/CD pipeline (Github Actions) to publish resume in different themes, formats, languages
      \item This resume was created using pysira: hmiladhia.github.io/en or hmiladhia.github.io/resume.pdf
  \end{itemize}

\divider
\cvevent{\faCube \href{https://pypi.org/project/nbmanips/}{nbmanips: Split, merge and convert IPython Notebooks}}{\cvtag{Jupyter}\cvtag{Databricks}\cvtag{Zeppelin}}{        
      May 2021 - Present
      }{}

An open-source package containing a collection of tools to manipulate Notebooks via a python or CLI.

\begin{itemize}
      \item Creation of a python package to manipulate (Split, Merge, ...) Notebooks
      \item Use of CI/CD piplines (Github Actions) to Lint, Test and publish the package to Pypi
  \end{itemize}

\divider
\cvevent{\faLaptop \href{https://spot-language.onrender.com/}{Spot-Language: Programming Language Detection}}{\cvtag{Classification}\cvtag{Flask}\cvtag{NLP}\cvtag{Lime}}{        
    January 2020
  }{}

Spot-Language is a classification model to detect the language used in code snippets.

\begin{itemize}
      \item Construction of a Training dataset from github public repositories.
      \item Building an ML Experiment to train an NLP model (Random Forest, Neural Networks, ...) for language classification.
      \item Deployment of Web Application to serve as demo of the ML Model.
  \end{itemize}



%%%%%%%%%%%%%%%%%%%%%%%%%%%%%%%%%%%%%%%%%%%%%%%%%%%%%%%%%
%% Switch to the right column. This will now automatically move to the second
%% page if the content is too long.
\switchcolumn
\cvsection{About}
{\small
I'm a passionate Data Scientist and a programming enthusiast. I've worked on fraud/anomaly detection and document classification problems
}

\cvsection{Education}

\cvevent{Engineering double degree}{\cvpart{\faUserGraduate ENSTA ParisTech }\cvpart{\faUserGraduate  ENIT}}{2016 - 2019}{France - Tunisia}
{\small Specialized in \textbf{Artificial Intelligence}}

\divider
\cvevent{Preparatory School}{\cvpart{\faUserGraduate IPEIT }\cvpart{\faUserGraduate  Esprit Prépa}}{2014 - 2015}{France - Tunisia}
{\small Specialized in \textbf{Mathematics \& Physics}}
{\small Ranked 10th in the entrance exam for engineering schools among 3,000 candidates.}



\cvsection{Skills}

\cvsubsection{Machine Learning}
\cvtag{ scikit-learn}
\cvtag{ Tensorflow/keras}
\cvtag{ trax}
\cvtag{ NLP}
\cvtag{ nltk}
\cvtag{ spaCy}
\cvtag{ Gensim}
\cvtag{ pyod}

\divider
\cvsubsection{Data Wrangling}
\cvtag{ pandas}
\cvtag{ pyspark}
\cvtag{\faDatabase SQL}
\cvtag{ ETL}

\divider
\cvsubsection{Programming Languages}
\cvtag{\faPython Python}
\cvtag{ C/C++}
\cvtag{ C\#}
\cvtag{\faJs JavaScript}

\divider
\cvsubsection{MLOps}
\cvtag{\faGithub Github Workflows}
\cvtag{\faMicrosoft Azure Pipelines}
\cvtag{\faGitSquare Git}
\cvtag{\faLinux Linux}
\cvtag{\faDocker Docker}
\cvtag{\faMicrosoft Azure ML}
\cvtag{ Databricks}
\cvtag{ huggingface}

\divider
\cvsubsection{Web Scraping}
\cvtag{ selenium}
\cvtag{ requests}
\cvtag{ BeautifulSoup4}

\divider
\cvsubsection{Web}
\cvtag{ HTML}
\cvtag{ CSS}
\cvtag{\faJs JavaScript}
\cvtag{ fastapi}
\cvtag{\faFlask flask}
\cvtag{ Dash}



\cvsection{Languages}

\cvskill{\large Arabic}{6}
\cvskill{\large English}{5}
\cvskill{\large French}{5}


\cvsection{Certificates}

\cvachievement{\faGraduationCap}{\href{https://www.coursera.org/account/accomplishments/specialization/certificate/J5FL39Y2CK4D}{Cloud-Native Development with OpenShift and Kubernetes Specialization}}{\cvpart{Coursera}\cvpart{\faCalendar 2023 2023}}
\cvachievement{\faGraduationCap}{\href{https://www.coursera.org/account/accomplishments/specialization/certificate/6XMVEAR45K6G}{Machine Learning Engineering for Production (MLOps)}}{\cvpart{Coursera}\cvpart{\faCalendar 2022 2022}}
\cvachievement{\faGraduationCap}{\href{https://triplebyte.com/tb/dhia-hmila-hatf9ne/certificate}{Machine Learning \& General Coding Logic}}{\cvpart{Triplebyte}\cvpart{\faCalendar 2023 2023}}

\end{paracol}

\end{document}