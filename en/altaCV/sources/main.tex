%%%%%%%%%%%%%%%%%
% This is an example CV created using altacv.cls (v1.3, 10 May 2020) written by
% LianTze Lim (liantze@gmail.com), based on the
% Cv created by BusinessInsider at http://www.businessinsider.my/a-sample-resume-for-marissa-mayer-2016-7/?r=US&IR=T
%
%% It may be distributed and/or modified under the
%% conditions of the LaTeX Project Public License, either version 1.3
%% of this license or (at your option) any later version.
%% The latest version of this license is in
%%    http://www.latex-project.org/lppl.txt
%% and version 1.3 or later is part of all distributions of LaTeX
%% version 2003/12/01 or later.
%%%%%%%%%%%%%%%%

%% If you are using \orcid or academicons
%% icons, make sure you have the academicons
%% option here, and compile with XeLaTeX
%% or LuaLaTeX.
% \documentclass[10pt,a4paper,academicons]{altacv}

%% Use the "normalphoto" option if you want a normal photo instead of cropped to a circle

\documentclass[10pt,letter,ragged2e,withhyper]{altacv}

%% AltaCV uses the fontawesome5 and academicon fonts and packages.
%% See http://texdoc.net/pkg/fontawesome5 and http://texdoc.net/pkg/academicons for full list of symbols.
%% You MUST compile with XeLaTeX or LuaLaTeX if you want to use academicons.

% Change the page layout if you need to
\geometry{left=1cm,right=1cm,top=0.8cm,bottom=0.8cm,columnsep=0.75cm}

% The paracol package lets you typeset columns of text in parallel
\usepackage{paracol}


% Change the font if you want to, depending on whether
% you're using pdflatex or xelatex/lualatex
\ifxetexorluatex
  % If using xelatex or lualatex:
  \setmainfont{Lato}
\else
  % If using pdflatex:
  \usepackage[default]{lato}
\fi

% Change the colours if you want to
\definecolor{VividPurple}{HTML}{3E0097}
\definecolor{SlateGrey}{HTML}{2E2E2E}
\definecolor{LightGrey}{HTML}{666666}
\definecolor{LimeGreen}{HTML}{11710E}
\definecolor{Blue}{HTML}{253489}
\definecolor{Teal}{HTML}{109EAA}
\definecolor{Silver}{HTML}{B5B5B5}
\definecolor{DarkPastelRed}{HTML}{450808}
\definecolor{PastelRed}{HTML}{8F0D0D}
\definecolor{GoldenEarth}{HTML}{E7D192}
\definecolor{DarkGreen}{HTML}{303d36}
\definecolor{MyGreen}{HTML}{88A05B}

\colorlet{name}{black}
\colorlet{tagline}{teal}
\colorlet{heading}{teal}
\colorlet{headingrule}{teal}
\colorlet{subheading}{DarkGreen}
\colorlet{accent}{teal}
\colorlet{emphasis}{SlateGrey}
\colorlet{body}{LightGrey}

% Change some fonts, if necessary

% Change the bullets for itemize and rating marker
% for \cvskill if you want to
\renewcommand{\itemmarker}{{\small\textbullet}}
\renewcommand{\ratingmarker}{\faCircle}

\renewcommand{\divider}{\textcolor{body!30}{\hdashrule{\linewidth}{0.6pt}{0.5ex}}\medskip}



%% sample.bib contains your publications

\begin{document}
\newcommand{\cvpart}[1]{\ifstrequal{#1}{}{}{{\small\makebox[0.5\linewidth][l]{~#1}}}}

\name{Dhia Hmila}
\tagline{Senior Data Scientist at AXA France}\photoR{3.5cm}{photo.png}
\photoL{3cm}{qrcode.png}

\personalinfo{%
  \email{dhiahmila@gmail.com}  \phone{+33610201606}  \location{92700, Colombes, France}  \homepage{hmiladhia.github.io}
    \printinfo{\faGithub}{hmiladhia}[https://github.com/hmiladhia]
    \printinfo{\faLinkedin}{dhia-hmila}[https://www.linkedin.com/in/dhia-hmila]
    \printinfo{\faMedium}{Dhia Hmila}[https://dhiahmila.medium.com/]
    \printinfo{\faStackOverflow}{dhia-hmila}[https://stackoverflow.com/users/8655480/dhia-hmila?tab=profile]
  
  % \printinfo{\faPaw}{Hey ho!}
}

\makecvheader

%% Depending on your tastes, you may want to make fonts of itemize environments slightly smaller
\AtBeginEnvironment{itemize}{\small}

%% Set the left/right column width ratio to 6:4.
\columnratio{0.6}

% Start a 2-column paracol. Both the left and right columns will automatically
% break across pages if things get too long.
\begin{paracol}{2}

\cvsection{Professional experience}

\cvevent{Senior Data Scientist}{Data Science Team, AXA France}{        
      June 2021 - Present
      }{Nanterre, France}

\begin{itemize}
  \item Automated the indexing of 70\% of AXA's incoming documents by developing a retrainable document classification AI System (exposed as an API).
  \item Developed text extraction AI Systems using computer vision to read documents (Driver's license, National ID card, etc.). The models processed over 6M documents in batches using PySpark Jobs
  \item Developed a range of reusable python packages for document processing (OCR, object detection, etc.) The packages are designed using the strategy pattern to quickly experiment with different algorithms / models.
  \item Created a YAML-based templating system to streamline the creation of Azure ML pipelines.
  \item Built reusable Azure ML Pipelines (Continuous Training) to automatically retrain AI Systems and publish to Model Registry.
  \item Designed Azure DevOps CI/CD/CT pipeline templates to be used in the Team's projects.
  \item Led the hiring process, onboarding new team members and facilitating efficient knowledge transfer.
  \item Coordinated the annotation process with annotators and SMEs (subject matter experts) to ensure consistent datasets across various projects.
  \item Led AXA's 'Python COP' initiative, organizing a series of monthly programming talks and events on topics such as testing, web scraping, packaging, code quality, etc.
  \item Teached a series of training sessions/courses on Python, Data Manipulation (pandas \& pyspark), Software Engineering best practices for over 30 AXA collaborators.
\end{itemize}

\divider
\cvevent{Graduate Data Scientist}{Data, Fraud, Waste \& Abuse Team, AXA France}{        
      May 2020 - June 2021
      }{Nanterre, France}

\begin{itemize}
  \item Engineered an advanced AI-driven Fraud, Waste, and Abuse Detection Tool with a wide-ranging impact across multiple scopes, including Claims, Health Providers, Beneficiaries, and more. This innovative tool played a pivotal role in achieving savings exceeding 150k euros in 2022.
  \item Designed and trained various fraud detection models for fraudulent patterns as part of the FWA detection tool.
\end{itemize}

\divider
\cvevent{Data Scientist}{Shift Technology}{        
      April 2019 - March 2020
      }{Paris, France}

\begin{itemize}
  \item Applied Natural Language Processing (NLP) techniques to cluster Claim managers' notes, leveraging advanced visualization methods to uncover and identify emerging fraud patterns.
  \item Built new AI Models for Shift's fraud detection solution \& Deployed their solution for a new customer.
\end{itemize}

\divider
\cvevent{Research Internship}{U2IS - ENSTA Paris}{        
      May 2018 - August 2018
      }{Palaiseau, France}

\begin{itemize}
  \item Created a synthetic image dataset by building a simulation of an agent in an in-house environment and using Raytracing to render realistic images.
  \item Studied Catastrophic Forgetting in Incremental Learning of Principal Components of generated Images
\end{itemize}



\cvsection{Projects}

\cvevent{\faCube \href{https://github.com/hmiladhia/mlflower}{mlflower: Lightweight orchestration tool for mlflow projects}}{\cvtag{mlflow}\cvtag{orchestration}\cvtag{experiment-tracking}}{        
      November 2023 - Present
      }{}

Tool to extend mlflow projects with inbuilt multi step workflow orchestration


\divider
\cvevent{\faCube \href{https://github.com/hmiladhia/pysira}{pysira: publish resumes in different formats}}{\cvtag{Github Actions}\cvtag{Jinja2 Templates}\cvtag{jsonschema}}{        
      February 2023 - Present
      }{}

CLI tool to export 'jsonresume' files to different formats (HTML, TeX, PDF, etc.) and languages.

\begin{itemize}
      \item Created a CI/CD pipeline (Github Actions) to publish resumes in different themes, formats and languages
      \item This resume was created using pysira: hmiladhia.github.io, hmiladhia.github.io/cv.pdf, hmiladhia.github.io/fr/kendall or hmiladhia.github.io/resume.pdf
  \end{itemize}

\divider
\cvevent{\faCube \href{https://pypi.org/project/nbmanips/}{nbmanips: Split, merge and convert IPython Notebooks}}{\cvtag{Jupyter}\cvtag{Databricks}\cvtag{Zeppelin}}{        
      May 2021 - Present
      }{}

An open-source package containing a collection of tools to manipulate Notebooks via a python or CLI.

\begin{itemize}
      \item Developed a python package/CLI to manipulate (Split, Merge, ...) Jupyter Notebooks
      \item Used CI/CD piplines (Github Actions) to Lint, Test and publish the package to Pypi
  \end{itemize}



%%%%%%%%%%%%%%%%%%%%%%%%%%%%%%%%%%%%%%%%%%%%%%%%%%%%%%%%%
%% Switch to the right column. This will now automatically move to the second
%% page if the content is too long.
\switchcolumn
\cvsection{About}
{\small
I am a passionate Data Scientist and programming enthusiast who has tackled challenges 
in fraud/anomaly detection and document classification. 
I am also an active contributor on GitHub.

}

\cvsection{Education}

\cvevent{Engineering double degree}{\cvpart{\faUserGraduate ENSTA ParisTech }\cvpart{\faUserGraduate  ENIT}}{2016 - 2019}{France - Tunisia}
{\small Specialized in \textbf{Artificial Intelligence}}

\divider
\cvevent{Preparatory School}{\cvpart{\faUserGraduate IPEIT }\cvpart{\faUserGraduate  Esprit Prépa}}{2014 - 2015}{France - Tunisia}
{\small Specialized in \textbf{Mathematics \& Physics}}
{\small Ranked 10th in the entrance exam for engineering schools among 3,000 candidates.}



\cvsection{Skills}

\cvsubsection{Machine Learning}
\cvtag{ mlflow}
\cvtag{ scikit-learn}
\cvtag{ Tensorflow}
\cvtag{ pytorch}
\cvtag{ trax}
\cvtag{ pyod}
\cvtag{ transformers}
\cvtag{ huggingface}
\cvtag{ pytorch-lightning}

\divider
\cvsubsection{Computer Vision}
\cvtag{ OpenCV}
\cvtag{ torchvision}
\cvtag{ Vision Transformers}
\cvtag{ VGG16}
\cvtag{ YOLO}
\cvtag{ Feature Matching}
\cvtag{ tesseract}
\cvtag{ easy-ocr}
\cvtag{ paddleOCR}

\divider
\cvsubsection{Natural Language Processing}
\cvtag{ nltk}
\cvtag{ spaCy}
\cvtag{ Gensim}
\cvtag{ torchtext}
\cvtag{ langchain}

\divider
\cvsubsection{Data Wrangling}
\cvtag{ pandas}
\cvtag{ pyspark}
\cvtag{\faDatabase SQL}
\cvtag{ ETL}

\divider
\cvsubsection{Programming Languages}
\cvtag{\faPython Python}
\cvtag{ C/C++}
\cvtag{ C\#}
\cvtag{\faJs JavaScript}

\divider
\cvsubsection{MLOps}
\cvtag{\faGithub Github Workflows}
\cvtag{\faMicrosoft Azure Pipelines}
\cvtag{\faGitSquare Git}
\cvtag{\faLinux Linux}
\cvtag{\faDocker Docker}
\cvtag{ Kubernetes}

\divider
\cvsubsection{Cloud}
\cvtag{\faMicrosoft Azure ML}
\cvtag{ Databricks}
\cvtag{ Heroku}

\divider
\cvsubsection{Web Scraping}
\cvtag{ selenium}
\cvtag{ requests}
\cvtag{ BeautifulSoup4}

\divider
\cvsubsection{Web}
\cvtag{ HTML}
\cvtag{ CSS}
\cvtag{\faJs JavaScript}
\cvtag{ fastapi}
\cvtag{\faFlask flask}
\cvtag{ Dash}



\cvsection{Languages}

\cvskill{\large Arabic}{6}
\cvskill{\large English}{5}
\cvskill{\large French}{5}
\cvskill{\large German}{1}


\cvsection{Certificates}

\cvachievement{\faGraduationCap}{\href{https://www.coursera.org/account/accomplishments/specialization/certificate/J5FL39Y2CK4D}{Cloud-Native Development with OpenShift and Kubernetes Specialization}}{\cvpart{Coursera}\cvpart{\faCalendar 2023 2023}}
\cvachievement{\faGraduationCap}{\href{https://www.coursera.org/account/accomplishments/specialization/certificate/6XMVEAR45K6G}{Machine Learning Engineering for Production (MLOps)}}{\cvpart{Coursera}\cvpart{\faCalendar 2022 2022}}
\cvachievement{\faGraduationCap}{\href{https://triplebyte.com/tb/dhia-hmila-hatf9ne/certificate}{Machine Learning \& General Coding Logic}}{\cvpart{Triplebyte}\cvpart{\faCalendar 2023 2023}}
\cvachievement{\faGraduationCap}{\href{}{TOEIC Listening and Reading Test (Score 985/990)}}{\cvpart{ETS Global}\cvpart{\faCalendar 2018 2018}}

\end{paracol}

\end{document}